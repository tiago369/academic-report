\begin{thesisresumo}

    O trabalho consiste na explanação das percepções acerca dos desafios passados. Assim conta como foi desenvolvido e as táticas utilizadas para a conclusão dessa atividade. Tem como objetivo a absorção de  conhecimentos com o intuito de realizar projetos de robótica. Para isso foram realizados estudos de uma gama de conhecimentos para poder realizar as atividades, sendo entre eles programação, simulação e \textit{ROS}.

%//todoo quais desafios passados? passados de presente? esse termo não está legal. talvez vc queira dizer desafios estabelecidos para o voluntário Tiago??? não sei 
%//todoo "Assim conta..." quem conta?
%//wow importante vc começar a mudar a forma como escreve, sempre use a terceira pessoa do singular de forma objetiva e impessoal
%//todoo o ROS na parte final do parágrafo não faz muito sentido. o que vc quiz dizer?


%Escreva aqui o resumo da disserta\c{c}\~ao, incluindo os contextos geral e espec\'ifico, dentro dos quais a pesquisa foi realizada, o objetivo da pesquisa, assun\c{c}\~ao filos\'ofica, os m\'etodos de pesquisa usados e as poss\'iveis contribui\c{c}\~oes que o que \'e proposto pode trazer \`a sociedade.

\ \\

% use de três a cinco palavras-chave

\textbf{Palavras-chave}: robótica, estudo, programação, simulação.

\end{thesisresumo}
