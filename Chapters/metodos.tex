\chapter{Desenvolvimento do projeto}
\label{chap:metod}

Durante esta seção sera descrito o processo de construção dos desafios, incluindo os estudos necessarios para cada um, o processo de criação e especificidades. Será apresentado estas características para cada um desafio.

\section{Desafio Workbooks Python}

Para realização deste desafio foi realizado a pesquisa sobre bibliotecas em python para cada tarefa propria e em seguida criação de scripts e testes. Para dessa forma comcluir o desafio.

\section{Desafio C++}

Tem como inicio o estudo da linguagem C++ através da Code Academy com o seu curso. Depois, para cada tarefa foi realizado uma pesquisa propria, sendo sobre matematica teoria ou bibliotecas dessa linguagem. Assim, foi realizado o desafio.

\section{Turtlesim setpoint position}

O desafio se iniciou com um estudo sobre o ROS, mais especificamente sobre publisher e subscriber. Após isso, foi feito uma pesquisa acerca das tecnicas de navegação chegando no dead reckoninig. Em seguida, iniciou-se a fase de estudo da programação e testes do package ate chegar no produto final.

\section{Desafio Webots}

Em primeira etapa foram realizados os tutoriais do Webots, logo em diante foi feito um estudo da programação nesta plataforma. Depois disso foram realizados seguidos testes e alterações ate que o robo conseguisse se locomover corretamente para apos isso implementar o sensor de luz.

\section{Desafio Husky}

O desafio foi realizado com o uso do ROS Noetic. Para ser realizado foi feito \textit{git clone} de um repositorio no github do Husky em um workspace. Com isso, foi realizados os desafios porem alguns erros surgiram, mas foram solucionados atraves de pesquisa e ajuda de colegas.


% %--------- NEW SECTION ----------------------
% \section{Interface do Usuário}
% \label{sec:ui}
% \lipsum[1]

% %--------- NEW SECTION ----------------------
% \section{Simulação do sistema}
% \label{sec:sim}
% \lipsum[2-4]

