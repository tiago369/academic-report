\chapter{Resultados}
\label{chap:result}
Depois de finalizados os desafios foram coletados os seus resultados e disponibilizados no github.

\section{Desafio Workbooks Python}

Com a conclusão do desafio, os codigos foram testados e tiveram sucesso no seu funcionamento. Assim, 

\section{Desafio C++}

Os códigos foram compilados e executados obtendo sucesso no seu funcionamento. O primeiro

\section{Turtlesim setpoint position}

Depois de finalizado o package tinha seu funcionamento como exigia o regulamento. Portanto as coordenadas eram digitadas e a tartaruga realizava um deslocamento ate alcançar uma distância de pelo menos 0.1 do objetivo e assim encerrava seu programa.

\section{Desafio Webots}

Após o explicitado na metodologia o robô piooner3x conseguiu seguir o percurso sem grandes problemas. Desse modo e parar quando encontrava uma fonte luminosa, sendo no caso uma luminária em um canto do mapa dentro de XX segundos

\section{Desafio Husky}

As diferentes navegações foram corretamente executadas e colocadas em prática dentro dos simuladores gazebosim e rviz. Com isso foi capaz perceber como as diferentes técnicas de navegação aplicadas e combinadas afetam o deslocamento do ugv



