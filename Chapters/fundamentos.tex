\chapter{Conceito do projeto}
\label{chap:fundteor}
%--------- NEW SECTION ----------------------
Os robôs móveis têm a capacidade de se moverem sem a assistência de um 
operador humano. Os mesmos podem ser classificados,quanto ao sistema de 
locomoção, como terrestres, aquáticos e aéreos. Os terrestres são subdivididos 
em robôs que possuem rodas, pernas (bípedes) ou esteiras (ref:Review ArticleA 
review of mobile robots:). Cada um desses métodos possuem características especifícas 
quanto ao movimento a ser realizado. Os bípedes, por exemplo, simulam um caminhar 
antropomórfico, semelhante aos humanos.
isso é igual <=  === <> #{ #( www
<| |>
===



O desenvolvimento deste projeto consiste em produzir um robô que possa caminhar 
sobre duas pernas. Além disso, o walker deve se locomover de forma autonôma a fim 
de realizar uma dada missão.

Neste capítulo serão abordados os requisitos do cliente, os requisistos técnicos, 
a missão do robô e a pesquisa por similares. 


\section{Desafio Workbooks Python}

O desafio de Workbooks utilizando a linguagem python foi composto de 16 desafios de 
programação utilizando a linguagem python.

\section{Desafio C++}
O desafio de C++ foi composto de três desafios de programação que utiliza a linguagem 
de programação C++.

\section{Turtlesim setpoint position}
Consiste em utilizar o software Turtlesim para através do ROS escolher 
uma posição e fazê-lo se deslocar até ela. A movimentação programada na 
Turtlesim se baseia em uma teoria principal, que é o
Dead reckoning. Dessa forma, o Dead reckoning consistem em calcular a 
distancia que falta do robô para determinando objetivo e incorporar esses
valores a sua position e velocidade. No caso da Turtlesim foi calculado
sua distancia usando o teormea de pitagoras e o angulo pelo qua a tartaruga
precisou virar pela trigonometria.

%imagem  https://en.wikipedia.org/wiki/Dead_reckoning

Na navegação do Turtlesim, o teorema de pitagoras é utilizado para calcular
a distancia entre a mesma e o objetivo definido durante cada instante. Para 
isso foi utilizado a seguinte formula 
$d = \sqrt{(x\textsubscript{f} - x\textsubscript{i})^2 + (y\textsubscript{f} - y\textsubscript{i})^2}$
. Nos quais x\textsubscript{f} e y\textsubscript{f} são as distancias finais
e x\textsubscript{i} e y\textsubscript{i} são as distancias da tartaruga 
em determinado instante.

Assim, para calcular quanto que o Turtlesim precisa girar faz o uso da 
trigonometria, para calcular o angulo do qual a Turtlesim esta deslocada 
em relação ao objetivo. Com isso, se faz o uso da formula de arctg na qual 
$\theta = \frac{y\textsubscript{f} - y\textsubscript{i}}{x\textsubscript{f} - x\textsubscript{i}} $
. Por fim, pegamos esse angulo e subtraimos do angulo atual da Turtlesim
para descobrir o quanto a tartaruga precisa rotacionar.

Para definir a velocidade da tartaruga multiplicamos a distancia por uma constante e
para descobrir o quanto ela precisa rotacionar multiplicamos a 
angulação por outra constante. Assim esses dados são publicados no topic
cmd\_vel para alterar a velocidade da turtle até que ela chegue no objetivo
especificado.

\section{Desafio Webots}

O Webots é uma plataforma opensource usada para simular robôs. 
Desse modo, o desafio consiste em utilizar essa plataforma para 
simular um robo chamado piooner3x, corringindo o codigo ja existente
e alterando ele para caso ele encontre uma luminaria pare de se locomover.

A partir disso foram realizados os tutorias dessa plataforma para 
ter conhecimento de como a utiliza-la e como simular os robos. 
Em seguida foi colocado um sensor de luz no piooner3x para que 
o mesmo tenha uma forma de detectar a luminosidade do ambiente
e seu codigo foi alterado para que se a leitura do sensor 
ultrapasse determinado valor ele pare


\section{Desafio Husky}
Husky é um veiculo UGV no qual atraves do ROS pode ser 
simulado em conjunto com o gazebosim e o rviz, sendo ambos
simualdores no qual o primeiro é voltado para o ambiente 
ao redor do husky e o segundo é como esse ugv percebe o mundo.
O desafio consiste em utilizar os simuladores para testar
diferentes formas de navegação com o husky.

O primeiro desafio é simular o husky com o package move base. 
Consiste em dar uma localização no mundo e ele ira tentar atingir 
esse objetivo. Caso o ugv identifique algum obstaculo ele ira desviar,
ou caso fique preso ira entrar em um processo chamado conservative reset
se parar de ficar preso voltara a navegação, caso não entrara em 
clearing rotation, se mesmo assim continuar preso iniciara um agressive reset
e continuando preso vai por fim fazer uma clearing rotation e se mesmo 
assim continuar preso vai abortar a ação. Como mostra na imagem

%imagem https://wiki.ros.org/move_base

O segundo desafio é o amcl demo, o qual é a junção do move base com o
amcl. Assim o amcl é um sistema probabilistico de localização do robô
o qual atraves de sensores de laser fazem o tracking da posisção do 
robo dentro de um mapa.

%imagem do simulador 

Gmapping demo é o terceiro desafio, ele é a junção do move base com o
gmapping. Esse package prove um SLAM (Simultaneos localization and mapping), 
baseado em sensores a laser. Com o gmapping é criado um mapa 2D do ambiente.
Como pode ser mostrado na imagem abaixo.

%imagem do mapa

Por ultimo foi realizado o frontier exploration demo, é composto 
pelo move base, gmapping, e o frontier exploration. Dessa forma,
o frontier em conjunto com esses packages para realizar a 
exploração de ambientes

%----------------------------------------------------------

%--------- NEW SECTION ----------------------


%---------------picture------------------------------------
% \begin{figure}
%     \centering
%     \subfigure[Figure A]{\label{fig:a}\includegraphics[width=60mm]{./lq}}
%     \subfigure[Figure B]{\label{fig:b}\includegraphics[width=60mm]{./lq}}
%     \subfigure[Figure C]{\label{fig:c}\includegraphics[width=\textwidth]{./lq}}
%     \caption{Three simple graphs}
%     \label{fig:three graphs}
% \end{figure}
%----------------------------------------------------------

% \begin{figure}
%     \centering
%     \begin{subfigure}[b]{0.3\textwidth}
%         \centering
%         \includegraphics[width=\textwidth]{./lq}
%         \caption{$y=x$}
%         \label{fig:y equals x}
%     \end{subfigure}
%     \hfill
%     \begin{subfigure}[b]{0.3\textwidth}
%         \centering
%         \includegraphics[width=\textwidth]{./lq}
%         \caption{$y=3sinx$}
%         \label{fig:three sin x}
%     \end{subfigure}
%     \hfill
%     \begin{subfigure}[b]{0.3\textwidth}
%         \centering
%         \includegraphics[width=\textwidth]{./lq}
%         \caption{$y=5/x$}
%         \label{fig:five over x}
%     \end{subfigure}
%        \caption{Three simple graphs}
%        \label{fig:three graphs}
% \end{figure}


% %--------- NEW SECTION ----------------------
% \section{Assunto 2}
% \label{sec:ass2}
% flkjasdlkfjasdlkfjs

% \begin{table}[h]
%     \begin{subtable}[h]{0.45\textwidth}
%         \centering
%         \begin{tabular}{l | l | l}
%         Day & Max Temp & Min Temp \\
%         \hline \hline
%         Mon & 20 & 13\\
%         Tue & 22 & 14\\
%         Wed & 23 & 12\\
%         Thurs & 25 & 13\\
%         Fri & 18 & 7\\
%         Sat & 15 & 13\\
%         Sun & 20 & 13
%        \end{tabular}
%        \caption{First Week}
%        \label{tab:week1}
%     \end{subtable}
%     \hfill
%     \begin{subtable}[h]{0.45\textwidth}
%         \centering
%         \begin{tabular}{l | l | l}
%         Day & Max Temp & Min Temp \\
%         \hline \hline
%         Mon & 17 & 11\\
%         Tue & 16 & 10\\
%         Wed & 14 & 8\\
%         Thurs & 12 & 5\\
%         Fri & 15 & 7\\
%         Sat & 16 & 12\\
%         Sun & 15 & 9
%         \end{tabular}
%         \caption{Second Week}
%         \label{tab:week2}
%      \end{subtable}
%      \caption{Max and min temps recorded in the first two weeks of July}
%      \label{tab:temps}
% \end{table}