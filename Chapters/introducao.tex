\chapter{Introdução}
\label{chap:intro}
A robótica é a area que cresce de maneira gigantesca no mercado atual \cite{Robótica93:online}. Devido a esse fator urge capacitar mão de obra no mercado. Assim foi criado o Centro de Competências de Robótica e sistemas autônomos, para permitir uma capacitação de profissionais nessa area.

Dessa forma, o Centro de Competências elabora atividades para poder treinar essa futura mão de obra. Assim surgem os desafios executados dentro deste relatório.

O seguinte relatório foi realizado para o Centro de Competências de Robótica e Sistemas Autônomos dentro do Centro Universitário Senai CIMATEC. Este tem por objetivo expor os desafios de capacitação realizados para poder atuar em atividades dentro do laboratório. Sendo realizados em um período de 2 meses

%//todoo isto não é um artigo é um relatório
%//wow parece que vc copiou os textos de algum lugar para jogar aqui , parece que falta algo ou q não está acabado.


%--------- NEW SECTION ----------------------
\section{Objetivos}
\label{sec:obj}
Os objetivos são realizar desafios de robótica, programação e simulação com a função de adquirir conhecimentos para poder atuar dentro do laboratório.
\label{sec:obj}

\subsection{Objetivos Específicos}
\label{ssec:objesp}
%//wow é sempre importante acentuar as palavras há várias q ficaram sem acentuação, como laboratório, autônomos...
Os objetivos específicos deste projeto são:
\begin{itemize}
      \item Aprender ROS e como utiliza-lo;
      \item Desenvolver habilidades de programação em Python e C++;
      \item Obter conhecimento de simulação e programação de robôs usando o Webots;
      \item Criar um package no ROS para controlar o turtlesim;
      \item Obter conhecimentos a cerca de navegação usando o Husky
  \end{itemize}

%\subsubsection*{Objetivos específicos principais}
%\label{sssec:obj-principais}


%--------- NEW SECTION ----------------------
\section{Justificativa}
\label{sec:justi}

%Importancia do: python para a robotica, c++, webots, ros, navegaçao do husky

Esse trabalho tem como função de capacitar o estudante acerca das ferramentas necessárias para trabalhar com robótica, sendo aprendendo a utilizar o framework ROS com a versão noetic, ou habilitando seu conhecimento em programação.

A programação é a base para o desenvolvimento de robôs. Desse modo urge utilizar linguagens que facilitam essa produção. Segundo \cite{WhenToUs14:online} o python tem uma função muito importante na robótica, que se deve graças a facilidade de escrita da sua linguagem. Ja o C++ tem um fator importante ligado a performance, ja que é uma linguagem de mais baixo nível.

Para poder desenvolver robôs é muito importante fazer sua simulação, testando sua programação e funcionamento. Assim, o Webots cumpre essa função sendo um simulador gratuito e opensource. Neste simulador, pode testar como o protótipo se comporta fisicamente no ambiente e se a programação está cumprindo o seu papel. Evitando que se construa um protótipo físico precocemente.

O uso de frameworks de robótica é indispensável para quem trabalha nessa area, visto que eles facilitam o trabalho de programação e simulação dos projetos desenvolvidos. Assim o foi usado o ROS noetic, sendo a versão mais recente do ROS 1, devido a sua popularidade. Como visto em \cite{ROSWhyRO33:online}.

%//todoo isso não é uma justificativa. Fale porque é importante fazer o seu objetivo, traga números que corroboram o seu pensamento, que justifique o porque disso

%--------- NEW SECTION ----------------------
\section{Organização do documento}
\label{section:organizacao}

Este documento apresenta $3$ capítulos e está estruturado da seguinte forma:

\begin{itemize}

  \item \textbf{Capítulo \ref{chap:intro} - Introdução}: Contextualiza o âmbito, no qual a pesquisa proposta está inserida. Apresenta, portanto, a definição do problema, objetivos e justificativas da pesquisa e como este \thetypeworkthree está estruturado;
  \item \textbf{Capítulo \ref{chap:desenv} - Desenvolvimento}: Explicita com toda a base teorica como o trabalho foi desenvolvido, o processo pelo qual ele passou e o que foi obtido.
% \item \textbf{Capítulo \ref{chap:fundteor} - Fundamentação Teórica}: Explicita toda a base teórica com o qual o trabalho foi produzido, tratando das bases para a sua construção;
%  \item \textbf{Capítulo \ref{chap:metod} - Materiais e Métodos}: Evidencia por qual processo foi executado o objetivo desse artigo até a obtenção dos resultados;
%  \item \textbf{Capítulo \ref{chap:result} - Resultados}: Mostra o que foi obtido com a realização desse trabalho;
  \item \textbf{Capítulo \ref{chap:conc} - Conclusão}: Apresenta as conclusões, contribuições e algumas sugestões de atividades de pesquisa a serem desenvolvidas no futuro.

\end{itemize}
