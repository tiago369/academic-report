\chapter{Introdução}
\label{chap:intro}

Este artigo consiste na exposição dos desafios realizados para o 
Laboratório de Robótica e Sistemas Autônomos. Assim, esses desafios foram realizados ao longo de dois meses. Esses desafios tem como principal função realizar a capacitação para pode atuar dentro do laboratório


%--------- NEW SECTION ----------------------
\section{Objetivos}
\label{sec:obj}
Os objetivos são realizar desafios de robótica, programação 
e simulação com a função de adquirir conhecimentos para poder atuar dentro do laboratório.
\label{sec:obj}

\subsection{Objetivos Específicos}
\label{ssec:objesp}
Os objetivos específicos deste projeto são:
\begin{itemize}utilizá-lo
      \item Aprender ROS e como utiliza-lo;
      \item Desenvolver habilidades de programação em Python e C++;
      \item Obter conhecimento de simulação e programação de robôs usando o Webots;
      \item Criar um package no ROS para controlar o turtlesim;
      \item Obter conhecimentos a cerca de navegação usando o Husky
  \end{itemize}

%\subsubsection*{Objetivos específicos principais}
%\label{sssec:obj-principais}


%--------- NEW SECTION ----------------------
\section{Justificativa}
\label{sec:justi}

Esse trabalho tem como função de capacitar o estudante acerca das 
ferramentas necessarias para trabalhar com robótica, sendo aprendendo 
a utilizar o framework ROS com a versão noetic, ou habilitando seu
conhecimento em programação.


%--------- NEW SECTION ----------------------
\section{Organização do documento}
\label{section:organizacao}

Este documento apresenta $5$ capítulos e está estruturado da seguinte forma:

\begin{itemize}

  \item \textbf{Capítulo \ref{chap:intro} - Introdução}: Contextualiza o âmbito, no qual a pesquisa proposta está inserida. Apresenta, portanto, a definição do problema, objetivos e justificativas da pesquisa e como este \thetypeworkthree está estruturado;
  \item \textbf{Capítulo \ref{chap:fundteor} - Fundamentação Teórica}: Explicita toda a base teorica com o qual o trabalho foi produzido, tratando das bases para a sua construção;
  \item \textbf{Capítulo \ref{chap:metod} - Materiais e Métodos}: Evidencia por qual processo foi executado o objetivo desse artigo até a obtenção dos resultados;
  \item \textbf{Capítulo \ref{chap:result} - Resultados}: Mostra o que foi obtido com a realização desse trabalho;
  \item \textbf{Capítulo \ref{chap:conc} - Conclusão}: Apresenta as conclusóes, contribuições e algumas sugestões de atividades de pesquisa a serem desenvolvidas no futuro.

\end{itemize}
